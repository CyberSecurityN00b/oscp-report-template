% !TeX spellcheck = en_US
% -- Exercise questions --
\newcommand{\exerciseq}[1] {
	\textbf{#1}
}

% -- Screenshots/figures --
\newcommand{\screenshot}[2] {
	\begin{figure}[H]
			\centering
			\shadowbox{\includegraphics[width=.95\textwidth]{#1}}
			\captionsetup{justification=centering}
			\caption{#2}
			\label{fig:#1}
	\end{figure}
}

% -- Modified Exploits --
\newcommand{\exploitmodified}[6] {
	% #1 - Exploit Name
	% #2 - Exploit Author
	% #3 - Exploit Link
	% #4 - Exploit Description
	% #5 - Command Line
	% #6 - Changes and Notes
	\begin{tcolorbox}[breakable]
		\begin{center}
			\textbf{Modified Exploit Used}
		\end{center}
		\begin{tabularx}{\textwidth}{| l X |}
					\hline
					\textbf{Exploit} & #1 \\
					\textbf{Author} & #2 \\
					\textbf{Source} & \url{#3} \\
					\hline
				\end{tabularx}
				\newline
				\begin{tabularx}{\textwidth}{|| X ||}
					\hline
			\begin{minipage}[t]{.95\textwidth}
				\textbf{Description:} \hfill\break #4
			\end{minipage} \\
			\hline
			\ifx&#5&
			\else
				\begin{minipage}[t]{.95\textwidth}
					\textbf{Commands/Todos:} #5
				\end{minipage} \\
				\hline
			\fi
		\end{tabularx}
		\newline\newline
		\textbf{Explanation of Changes and Additional Notes:} #6
	\end{tcolorbox}
	\begin{center}
		\emph{Note: Lines highlighted in yellow below mark changes from the original source. For lines that are broken across multiple lines due to word-wrap, only the first line is highlighted.}
	\end{center}
}

% -- Unmodified Exploits --
\newcommand{\exploitunmodified}[5] {
	% #1 - Exploit Name
	% #2 - Exploit Author
	% #3 - Exploit Link
	% #4 - Exploit Description
	% #5 - Command Line
	% #6 - Changes and Notes
	\begin{tcolorbox}[breakable]
		\begin{center}
			\textbf{Unmodified Exploit Used}
		\end{center}
		\begin{tabularx}{\textwidth}{| l X |}
			\hline
			\textbf{Exploit} & #1 \\
			\textbf{Author} & #2 \\
			\textbf{Source} & \url{#3} \\
			\hline
		\end{tabularx}
		\newline
		\begin{tabularx}{\textwidth}{|| X ||}
			\hline
			\begin{minipage}[t]{.95\textwidth}
				\textbf{Description:} \hfill\break #4
			\end{minipage} \\
			\hline
			\ifx&#5&
			\else
				\begin{minipage}[t]{.95\textwidth}
					\textbf{Commands/Todos:} #5
				\end{minipage} \\
				\hline
			\fi
		\end{tabularx}
	\end{tcolorbox}
}

% -- "custom" exploits
\newcommand{\exploitcustom}[6] {
	% #1 - Exploit Name
	% #2 - Exploit Author
	% #3 - Exploit Link
	% #4 - Exploit Description
	% #5 - Command Line
	% #6 - Changes and Notes
	\begin{tcolorbox}[breakable]
		\begin{center}
			\textbf{Custom Exploit Used}
		\end{center}
		\begin{tabularx}{\textwidth}{| l X |}
					\hline
					\textbf{Exploit} & #1 \\
					\textbf{Reference} & \url{#3} \\
					\hline
				\end{tabularx}
				\newline
				\begin{tabularx}{\textwidth}{|| X ||}
					\hline
			\begin{minipage}[t]{.95\textwidth}
				\textbf{Description:} \hfill\break #4
			\end{minipage} \\
			\hline
			\ifx&#5&
			\else
				\begin{minipage}[t]{.95\textwidth}
					\textbf{Commands/Todos:} #5
				\end{minipage} \\
				\hline
			\fi
		\end{tabularx}
	\end{tcolorbox}
	\begin{center}
		\emph{Note: Custom exploits are not necessarily original exploits. They include exploits which \osid{} has ported from one language to another or significantly modified from the original.}
	\end{center}
}

% -- Highlight Line --
\newcommand{\highlightline}{
	\makebox[0pt][l]{\color{yellow}\rule[-1ex]{.99\linewidth}{3ex}}
}

% -- Report New Host --
\newcommand{\reportnewhost}[4]{
	% #1 - Hostname
	% #2 - IP Address
	% #3 - Operating System Name
	% #4 - Operating System Version
	
	\subsection{#1}
	
	\begin{tcolorbox}[title=Host Information,sharp corners=uphill]
		\begin{tabularx}{\textwidth}{| l | X |}
			\hline
			\textbf{Hostname} & #1 \\
			\hline
			\textbf{IP Address} & #2 \\
			\hline
			\ifx&#4&
				\textbf{Operating System} & #3 \\
			\else
				\multirow{2}{*}{\textbf{Operating System}} & #3 \\
				& \textit{#4} \\
			\fi
			\hline
		\end{tabularx}
	\end{tcolorbox}

	\subsubsection{Enumerated Ports}
	\begin{center}
	\small{\textit{The port(s) involved in acquiring remote access to the machine are highlighted in red.}}
	\end{center}
}

% -- TCP Ports
\newcommand{\reporttcpports}[1]{
	\begin{tcolorbox}[title=Enumerated TCP Ports,sharp corners=uphill]
		\begin{tabularx}{\textwidth}{| r | X |}
			\hline
			\textbf{Port} & \textbf{Service Information} \\
			\hline
			#1
			\hline
		\end{tabularx}
	\end{tcolorbox}
}

% -- UDP Ports
\newcommand{\reportudpports}[1]{
	\begin{tcolorbox}[title=Enumerated UDP Ports,sharp corners=uphill]
		\begin{tabularx}{\textwidth}{| r | X |}
			\hline
			\textbf{Port} & \textbf{Service Information} \\
			\hline
			#1
			\hline
		\end{tabularx}
	\end{tcolorbox}
}

\newcommand{\portservicevuln}[2]{
	\rowcolor{red!25} #1 & #2 \\
	\hline
}

\newcommand{\portservice}[2]{
	#1 & #2 \\
	\hline
}

\newcommand{\reportexploitsummarysection}{
	\subsubsection{Exploit Summary}
	\begin{center}
		\emph{Only the exploit(s) used to gain remote access to this host are summarized in this section.\newline{}This should not be considered a complete coverage of this host's vulnerabilities.\newline{}\textbf{Note:} Remediations are strictly recommendations to address the specific issue noted.}
	\end{center}
}

% -- Report Remote Access --
\newcommand{\reportexploitsummary}[6]{
	% #1 - Basic ID or Name
	% #2 - Severity
	% #3 - Vulnerability Description
	% #4 - Vulnerability Impact
	% #5 - Vulnerability Fix
	% #6 - Exploit Reference
	\begin{tcolorbox}[title=#1 Summary,sharp corners=uphill]
		\begin{tabularx}{\textwidth}{| r | X |}
			\hline
			\textbf{Severity} & #2 \\
			\hline
			\textbf{Description} & #3 \\
			\hline
			\textbf{Impact} & #4 \\
			\hline
			\textbf{Remediation} & #5 \\
			\hline
		\end{tabularx}
		\ifx&#6&
		\else
			\newline
			\begin{center}
				Please reference appendix \vref{exploit:#6} for detailed information regarding the exploit used for this finding.
			\end{center}
		\fi
	\end{tcolorbox}
}

\newcommand{\reportpreexploitation}{
	\subsubsection{Service Enumeration}
}

\newcommand{\reportexploiting}{
	\subsubsection{Exploitation}
}

% -- Report Privilege Escalation --
\newcommand{\reportprivesc}{
	\subsubsection{Privilege Escalation}
}

\newcommand{\reportprivescnotneeded}{
	\subsubsection{Privilege Escalation}
	No additional steps were needed to gain elevated privileges.
}

\newcommand{\reportinvestigation}{
	\subsubsection{Host Investigation}
}

\newcommand{\reportinvestigationnotneeded}{
	\subsubsection{Host Investigation}
	\osid{} did not identify any files or information which warrant inclusion here.
}

\newcommand{\reportcleanup}{
	\subsubsection{Cleanup Actions Needed}
}

\newcommand{\reportcleanupnotneeded}{
	\subsubsection{Cleanup Actions Needed}
	All necessary cleanup actions were performed by \osid{}. Offensive Security does not need to take any additional steps.
}

\newcommand{\reportrecommendations}{
	\subsubsection{Recommendations}
	Recommendations to mitigate the specific vulnerabilities or mis-configurations employed:
}

% -- Screenshot of Proof --
\newcommand{\screenshotproof}[1]{
	% #1 - IP Address
	\screenshot{#1-proof.png}{Proof of successful exploitation with elevated privileges.}
}
\newcommand{\screenshotprooflocal}[1]{
	% #1 - IP Address
	\screenshot{#1-localproof.png}{Proof of successful exploitation with non-elevated privileges.}
}

%-- Severity Identifiers --
\newcommand{\severitycritical}{
	\textbf{\textcolor{red}{Critical}}
}
\newcommand{\severityhigh}{
	\textbf{\textcolor{orange}{High}}
}
\newcommand{\severitymedium}{
	\textbf{\textcolor{yellow}{Medium}}
}

\newcommand{\reportnotice}[1]{
	\begin{tcolorbox}[enhanced,attach boxed title to top center={yshift=-3mm,yshifttext=-1mm},
	  colback=blue!5!white,colframe=blue!75!black,colbacktitle=red!80!black,
	  title=N O T I C E,fonttitle=\bfseries,
	  boxed title style={size=small,colframe=red!50!black}]
	  #1
	\end{tcolorbox}
}

\newcommand{\reportdatahost}[1]{
	\texttt{#1}
}

\newcommand{\reportdatatool}[1]{
	\texttt{#1}
}